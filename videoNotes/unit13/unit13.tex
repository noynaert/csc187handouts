
%!TEX program = xelatex
\documentclass[letterpaper,12pt]{exam}
\usepackage{../videoNotes}
\usepackage{xcolor}
\usepackage[dvipsnames]{xcolor}
\usepackage{soul}

\newcommand{\unit}{Unit 13}
\pagestyle{headandfoot}
\firstpageheader{CSC 187 \semester\ \  \unit}{}{Name: $\rule{6cm}{0.15mm}$}
\runningheader{CSC 187 \semester}{\unit}{Page \thepage\ of \numpages}
\firstpagefooter{}{}{}
\runningfooter{}{}{}

\begin{document}
\section*{\unit\_010 -- Args (Command Line Arguments)} 
\par{\fontfamily{qzc}\selectfont\textbf{Video Length 5:53}}
\begin{questions}
\begin{samepage}
    \question What are command line arguments?
    \vspace{5mm}
\end{samepage}
\begin{samepage}
    \question How do command line arguments get into a java program?
    \vspace{5mm}
\end{samepage}

%----------------------------------
\section*{\unit\_012 -- Simulating Command Line Arguments in IntelliJ} 
\par{\fontfamily{qzc}\selectfont\textbf{Video Length }}

\begin{samepage}
    \question How do you simulate Command Line Arguments in IntelliJ?
    \vspace{5mm}
\end{samepage}

\begin{samepage}
    \question What would you do in IntelliJ if you were using two or more different sets of command line arguments to test your program??
    \vspace{5mm}
\end{samepage}
%----------------------------------
\section*{\unit\_020 -- Access Modifiers} 
\par{\fontfamily{qzc}\selectfont\textbf{Video Length }}

\begin{samepage}
    \question What does each access modifier mean?
      \begin{itemize}
        \item public
        \vspace{5mm}
        \item protected
        \vspace{5mm}
        \item default
        \vspace{5mm}
        \item private
        \vspace{5mm}
       \end{itemize}
\end{samepage}

\begin{samepage}
    \question What is the difference between \texttt{\textbf{protected}} and the default level of access protection
    \vspace{5mm}
\end{samepage}

\begin{samepage}
    \question Can a class have \texttt{\textbf{protected}} as an access modifier?
    \vspace{5mm}
\end{samepage}

\rule{20mm}{0.15mm}Why can't a class be declared private?

%----------------------------------
\section*{\unit\_030 -- Static } 
\par{\fontfamily{qzc}\selectfont\textbf{Video Length }}
\begin{samepage}
    \question Suppose there is a class named \texttt{\textbf{Thing}}.  It has a public static field named \texttt{\textbf{constant}}.  How would you access constant from a class other than \texttt{\textbf{Thing}}?
    \vspace{5mm}
\end{samepage}


%----------------------------------
\section*{\unit\_035 -- A Promise Kept } 
\par{\fontfamily{qzc}\selectfont\textbf{Video Length }}

\begin{samepage}
    \question Explain what each word means in the following
    \begin{verbatim}
        public class Homework{
            public static void main(String[] args){

            }
        }
    \end{verbatim}
      \begin{itemize}
        \item \code{public} (on the class line). Also, what do you know about the file name?
        \vspace{5mm}
        \item \code{class}
        \vspace{5mm}
        \item \code{Homework}
        \vspace{5mm}
        \item \code{public} (on the main line)
        \vspace{5mm}
        \item \code{static}
        \vspace{5mm}
        \item \code{void}
        \vspace{5mm}
        \item \code{main}
        \vspace{5mm}
        \item \code{String[]}
        \vspace{5mm}
        \item \code{args}
        \vspace{5mm}
       \end{itemize}
\end{samepage}

\begin{samepage}
    \question Which two words in the above list could you as a programmer change? This was not answered in the video, but think about it.  There are two words that could be changed.
    \vspace{5mm}
\end{samepage}

%----------------------------------
\section*{\unit\_040 -- Setup} 
\par{\fontfamily{qzc}\selectfont\textbf{Video Length }}

\begin{samepage}
    \question I could have copied the package off to a new project.  What advantage did I get from using it in the same project as Homework12?
    \vspace{5mm}
\end{samepage}

\begin{samepage}
    \question I rewrote the printPresident() method from Homework12.  However, I could not use printPresident.  The first line of printPresident is listed below.  Why couldn't I just use Homework.printPresident(presidents,n)?
    \begin{verbatim}
        private static void printPresidents(President[] presidents,int n) {
    \end{verbatim}
    \vspace{5mm}
\end{samepage}
%----------------------------------

\section*{\unit\_045 -- Setup, Part 2} 
\par{\fontfamily{qzc}\selectfont\textbf{Video Length }}

\begin{samepage}
    \question Both the Homework12 class and the LinearSearch classes had a function called makeName.  They had the same signature. Why didn't I have a problem? 
    \vspace{5mm}
\end{samepage}
\begin{samepage}
    \question We didn't do this in the video, but assume that I now wanted to go into the Homework12 class and call makeName from the LinearSearch class.  What command would I use in Homework12 if I wanted to call the makeName() method in LinearSearch()?
    \vspace{5mm}
\end{samepage}

%----------------------------------
\section*{\unit\_050 -- Big O} 

Side note:  For people who know Python, I was using something called "Jupyter Notebook."  I did it in Visual Studio Code.  Jupyter notebooks are a very useful tool.  I suggest that anyone who knows Python learn about Jupyter notebooks.

\par{\fontfamily{qzc}\selectfont\textbf{Video Length }}
\begin{samepage}
    \question Why is $O(N)$ called "Linear?"
    \vspace{5mm}
\end{samepage}

\begin{samepage}
    \question Why is the growth rate only of concern when there are large values of N?
    \vspace{5mm}
\end{samepage}

\begin{samepage}
    \question What is $0(1)$ called?  What does it mean?  Does constant seem like a good name for this?
    \vspace{15mm}
\end{samepage}


%----------------------------------

\section*{\unit\_060 -- compareTo()} 
\par{\fontfamily{qzc}\selectfont\textbf{Video Length }}

\begin{samepage}
    \question How can a class advertise that it is able to use a compareTo method?
    \vspace{5mm}
\end{samepage}
\begin{samepage}
    \question Assume that we want to make the Place class implement compareTo.  Write the \texttt{\textbf{public class Place}} statement to advertise that the compareTo method is implemented in the Place class.
    \vspace{5mm}
\end{samepage}
\begin{samepage}
    \question For a place, the comparison is based on the state.  If the states are equal, then the comparison is based on the city. 
    \vspace{55mm}
\end{samepage}


%----------------------------------
\section*{\unit\_070 -- } 
\par{\fontfamily{qzc}\selectfont\textbf{Video Length }}

\begin{samepage}
    \question What is the disadvantage of using a for loop for the linear search?
    \vspace{5mm}
\end{samepage}
\begin{samepage}
    \question Write the find method needed to do a comparison based on the birthPlace (in other words, find will return an entry of someone born in a given place.)
    \vspace{55mm}
\end{samepage}


%----------------------------------
\section*{\unit\_080 Part 1 -- YouTube video on Selection sort } 
\par{\fontfamily{qzc}\selectfont\textbf{Video Length }}

I hate tracing sorts.  There are a number of YouTube videos on the subject.  We will focus on the Selection Sort in this class.

Watch the YouTube video at https://youtu.be/EwjnF7rFLns?si=ZpQsuO9slKj7Pdgd.  I will put the URL in the Notes page.

If you don't want to use the notes, you can search YouTube for "Learn Selection Sort in 8 minutes" in the "Bro Code" channel.

\begin{samepage}
    \question The video talks in terms of "iterations."  What is true at the end of the first iteration?
    \vspace{5mm}
\end{samepage}
\begin{samepage}
    \question What is true at the end of the 2nd iteration?
    \vspace{5mm}
\end{samepage}

\begin{samepage}
    \question When he gets to the last item in the array, why doesn't he have to do an iteration for the last position?
    \vspace{5mm}
\end{samepage}
\begin{samepage}
    \question In the video he shows that the growth rate is $O(n^2)$.  What is going to happen to the time as n gets large.  (In practical terms, "Large" for an $n^2$ sort is in the low hundreds). 
    \vspace{5mm}
\end{samepage}
\begin{samepage}
    \question In the space below, write the three lines that do the swap.  Note that they are the last three lines inside the outer for() loop.
    \vspace{25mm}
\end{samepage}



%----------------------------------
\section*{\unit\_080 Part2 Sorting Strings } 
\par{\fontfamily{qzc}\selectfont\textbf{Video Length }}
\begin{samepage}
    \question In the video I changed "array" to "words."  That was a trivial change.  I added a parameter n because that is more realistic than using length.  I also changed "min" to "smallestSoFar."  I changed some data types from int to String.  What is the biggest non-trivial change I made in the code that was made necessary by changing from comparing ints to comparing Strings?
    \vspace{5mm}
\end{samepage}

%%%%%%%%%%%%%%%%%%%%%%%%%%%%%%%%%%%%%%%%%%%%%%%%%%%%%
\begin{center}
    \rule{0.5\textwidth}{.4pt}
\end{center}
Please write any lingering questions you have here.
\end{questions}

\end{document}