%!TEX program = xelatex
\documentclass[letterpaper,12pt]{exam}
\usepackage{../videoNotes}
\usepackage{xcolor}
\usepackage[dvipsnames]{xcolor}


\newcommand{\unit}{Unit 07}
\pagestyle{headandfoot}
\firstpageheader{CSC 187 \semester\ \  \unit}{}{Name: $\rule{6cm}{0.15mm}$}
\runningheader{CSC 187 \semester}{\unit}{Page \thepage\ of \numpages}
\firstpagefooter{}{}{}
\runningfooter{}{}{}

\begin{document}
\section*{\unit\_010 -- If Statement} % Unnumbered section
\par{\fontfamily{qzc}\selectfont\textbf{Video Length 12:30}}

\begin{questions}
\begin{samepage}
    \question What goes inside the (parenthesis) of an if statement?
    \vspace{5mm}
\end{samepage}
\begin{samepage}
    \question What goes inside the \{curly braces\} of an if statement?
    \vspace{5mm}
\end{samepage}
\begin{samepage}
    \question Are the \{curly braces\} always required?  What is the rule about whether they are always required?
    \vspace{5mm}
\end{samepage}
\begin{samepage}
    \question Does indentation matter to the Java compiler?  Why bother to indent?
    \vspace{5mm}
\end{samepage}

\begin{samepage}
    \question What is the danger of skipping the use of the curly braces?
    \vspace{5mm}
\end{samepage}
\begin{samepage}
    \question Assume there is a variable called \texttt{\textbf{temperature}} that holds the current temperature.  Write a statement that will print "It is freezing" if the temperature is below freezing on whatever temperature scale you wish to use.
    \vspace{40mm}
\end{samepage}

*{\unit\_020 -- If ... Else Statement} % Unnumbered section
\par{\fontfamily{qzc}\selectfont\textbf{Video Length 14:02}}

\begin{samepage}
    \question When does the "else" clause execute?
    \vspace{5mm}
\end{samepage}

\begin{samepage}
    \question True or false:  In an if...else statement, either the then or the else clause always executes.
    \vspace{5mm}
\end{samepage}

\begin{samepage}
    \question Does the else clause of an if...else need to have curly braces?
    \vspace{5mm}
\end{samepage}

\begin{samepage}
    \question Assume there is a variable called \texttt{\textbf{temperature}} that holds the current temperature.  Write a statement that will print "It is freezing" if the temperature is below freezing.  Write "Your plants are safe" if the temperature is not below freezing.  Be sure to use proper indentation and capitalization.
    \vspace{5mm}
\end{samepage}


*{\unit\_025 -- Conditional Statement} % Unnumbered section
\par{\fontfamily{qzc}\selectfont\textbf{Video Length 9:40}}

\begin{samepage}
    \question True or False, The conditional expression should only be used in situations where the expression returns a value.
    \vspace{5mm}
\end{samepage}

\begin{samepage}
    \question True or False, The conditional expression should be used in situations where it prints one of two alternative statements.
    \vspace{5mm}
\end{samepage}

\begin{samepage}
    \question Rewrite the following as an if...else statement.
    \begin{verbatim}
       message = (isRaining) ? "Inside" : "Outside";
    \end{verbatim}
    \vspace{35mm}
\end{samepage}

\section*{\unit\_030 -- Style} % Unnumbered section
\noindent \textbf{Video Length 7:30}


\begin{samepage}
    \question Rewrite the following code in better style.
    \begin{verbatim}
        if(temperatureOK == true)\{
    \end{verbatim}    
    \vspace{5mm}
\end{samepage}


\begin{samepage}
    \question Rewrite the following code in better style.
    \begin{verbatim}
        if(temperatureOK == false)\{
    \end{verbatim}    
    \vspace{5mm}
\end{samepage}


\begin{samepage}
    \question Rewrite the following code in better style.
    \begin{verbatim}
        if(temperature >= 70 && temperature <=100)
            swimmingOK = true;
        else
            swimmingOK = false;
    \end{verbatim}    
    \vspace{5mm}
\end{samepage}

\begin{samepage}
    \question What is unit testing?
    \vspace{5mm}
\end{samepage}

\section*{\unit\_040 -- Comparing Strings} % Unnumbered section
\noindent \textbf{Video Length: 15:36}

\section*{\unit\_050 -- It's about time} % Unnumbered section
\noindent \textbf{Video Length: 17:11}

\begin{samepage}
    \question What are some reasons that dates and times are hard?
    \vspace{25mm}
\end{samepage}

\begin{samepage}
    \question How does Java store times and dates?
    \vspace{5mm}
\end{samepage}

\begin{samepage}
    \question Define the following terms
      \begin{itemize} 
        \item Epoch
        \vspace{5mm}
        \item GMT
        \vspace{5mm} 
        \item UTC
        \vspace{5mm} 
        \item millisecond 
        \vspace{5mm} 
        \item nanosecond
       \end{itemize}
\end{samepage}

\begin{samepage}
    \question How long is each of the following in seconds?
      \begin{itemize}
        \item 1ms
        \vspace{5mm}
        \item 500 ms
        \vspace{5mm}
        \item 1000 ms
        \vspace{5mm}
        \item 6000 ms
        \vspace{5mm}
       \end{itemize}
\end{samepage}


\section*{\unit\_060 -- Dates and Times in Java} % Unnumbered section
\noindent \textbf{Video Length: }
\section*{\unit\_070 -- Swapping} % Unnumbered section
\noindent \textbf{Video Length: 5:34}

\begin{samepage}
    \question Assume that there are two String variables \texttt{\textbf{person1}} and \texttt{\textbf{person2}}. Write the code needed to "swap" the names, so that person1 now has the name that used to be in person2, and person2 now holds the name that used to be assigned to person1.  
    \vspace{5mm}
\end{samepage}


\section*{\unit\_080 Nested If} % Unnumbered section
\noindent \textbf{Video Length: }

%%%%%%%%%%%%%%%%%%%%%%%%%%%%%%%%%%%%%%%%%%%%%%%%%%%%%%%%%%%%%%%%%%%%%%%%%%%%
\begin{samepage}
    \begin{center}
    \rule{0.8\textwidth}{.4pt}
    \end{center}
	\question Comment or ask questions here if there are any issues you would like for the instructor to address.  It is not necessary to answer if you do not have questions.
	\vspace{30mm}
\end{samepage}

\end{questions}

\end{document}