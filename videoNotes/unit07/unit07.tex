%!TEX program = xelatex
\documentclass[letterpaper,12pt]{exam}
\usepackage{../videoNotes}
\usepackage{xcolor}
\usepackage[dvipsnames]{xcolor}


\newcommand{\unit}{Unit 07}
\pagestyle{headandfoot}
\firstpageheader{CSC 187 \semester\ \  \unit}{}{Name: $\rule{6cm}{0.15mm}$}
\runningheader{CSC 187 \semester}{\unit}{Page \thepage\ of \numpages}
\firstpagefooter{}{}{}
\runningfooter{}{}{}

\begin{document}
\section*{\unit\_010 -- If Statement} % Unnumbered section
\par{\fontfamily{qzc}\selectfont\textbf{Video Length 12:30}}

\begin{questions}
\begin{samepage}
    \question What goes inside the (parenthesis) of an if statement?
    \vspace{5mm}
\end{samepage}
\begin{samepage}
    \question What goes inside the \{curly braces\} of an if statement?
    \vspace{5mm}
\end{samepage}
\begin{samepage}
    \question Are the \{curly braces\} always required?  What is the rule about whether they are always required?
    \vspace{5mm}
\end{samepage}
\begin{samepage}
    \question Does indentation matter to the Java compiler?  Why bother to indent?
    \vspace{5mm}
\end{samepage}

\begin{samepage}
    \question What is the danger of skipping the use of the curly braces?
    \vspace{5mm}
\end{samepage}
\begin{samepage}
    \question Assume there is a variable called \texttt{\textbf{temperature}} that holds the current temperature.  Write a statement that will print "It is freezing" if the temperature is below freezing on whatever temperature scale you wish to use.
    \vspace{40mm}
\end{samepage}

*{\unit\_020 -- If ... Else Statement} % Unnumbered section
\par{\fontfamily{qzc}\selectfont\textbf{Video Length 14:02}}

\begin{samepage}
    \question When does the "else" clause execute?
    \vspace{5mm}
\end{samepage}

\begin{samepage}
    \question True or false:  In an if...else statement, either the then or the else clause always executes.
    \vspace{5mm}
\end{samepage}

\begin{samepage}
    \question Does the else clause of an if...else need to have curly braces?
    \vspace{5mm}
\end{samepage}

\begin{samepage}
    \question Assume there is a variable called \texttt{\textbf{temperature}} that holds the current temperature.  Write a statement that will print "It is freezing" if the temperature is below freezing.  Write "Your plants are safe" if the temperature is not below freezing.  Be sure to use proper indentation and capitalization.
    \vspace{5mm}
\end{samepage}


*{\unit\_025 -- Conditional Statement} % Unnumbered section
\par{\fontfamily{qzc}\selectfont\textbf{Video Length 9:40}}

\begin{samepage}
    \question True or False, The conditional expression should only be used in situations where the expression returns a value.
    \vspace{5mm}
\end{samepage}

\begin{samepage}
    \question True or False, The conditional expression should be used in situations where it prints one of two alternative statements.
    \vspace{5mm}
\end{samepage}

\begin{samepage}
    \question Rewrite the following as an if...else statement.
    \begin{verbatim}
       message = (isRaining) ? "Inside" : "Outside";
    \end{verbatim}
    \vspace{35mm}
\end{samepage}

\section*{\unit\_030 -- Style} % Unnumbered section
\noindent \textbf{Video Length 7:30}


\begin{samepage}
    \question Rewrite the following code in better style.
    \begin{verbatim}
        if(temperatureOK == true)\{
    \end{verbatim}    
    \vspace{5mm}
\end{samepage}


\begin{samepage}
    \question Rewrite the following code in better style.
    \begin{verbatim}
        if(temperatureOK == false)\{
    \end{verbatim}    
    \vspace{5mm}
\end{samepage}


\begin{samepage}
    \question Rewrite the following code in better style.
    \begin{verbatim}
        if(temperature >= 70 && temperature <=100)
            swimmingOK = true;
        else
            swimmingOK = false;
    \end{verbatim}    
    \vspace{5mm}
\end{samepage}

\begin{samepage}
    \question What is unit testing?
    \vspace{5mm}
\end{samepage}

\section*{\unit\_040 -- Comparing Strings} % Unnumbered section
\noindent \textbf{Video Length: 15:36}

\begin{samepage}
    \question Suppose there are two string variables named s1 and s2.  Write the code that will print "Same" if they are equal.  Ignore the case.  You do not have to do anything if they are equal.
    \vspace{15mm}
\end{samepage}

\begin{samepage}
    \question Suppose there are two string variables named s1 and s2.  Write the code that will print "Different" if they are equal.  Ignore the case.  DO NOT WRITE ANYTHING IF THEY ARE EQAUL.  (Hint: ! is probably your friend here).
    \vspace{15mm}
\end{samepage}

\begin{samepage}
    \question What does compareTo return?  (There are three answers)
    \vspace{5mm}
\end{samepage}

\begin{samepage}
    \question Suppose there are two string variables named s1 and s2.  Write the code that will print "S1 is smaller" if s1 would come alphabetically before s2.   Ignore the case.  You do not have to do anything if they are equal or s2 is less than s1.
    \vspace{5mm}
\end{samepage}


\section*{\unit\_050 -- It's about time} % Unnumbered section
\noindent \textbf{Video Length: 17:11}

\begin{samepage}
    \question What are some reasons that dates and times are hard?
    \vspace{25mm}
\end{samepage}

\begin{samepage}
    \question Write July 4, 2038 using ISO standard notation.
    \vspace{5mm}
\end{samepage}

\begin{samepage}
    \question What are two advantages of ISO standard dates over US and European date formats?
      \begin{enumerate}
        \vspace{5mm}
        \item 
        \vspace{5mm}
        \item 
        \vspace{5mm}
       \end{enumerate}
\end{samepage}


\begin{samepage}
    \question How does Java store times and dates?
    \vspace{5mm}
\end{samepage}

\begin{samepage}
    \question Define the following terms
      \begin{itemize} 
        \item Epoch
        \vspace{5mm}
        \item GMT
        \vspace{5mm} 
        \item UTC
        \vspace{5mm} 
        \item millisecond 
        \vspace{5mm} 
        \item nanosecond
       \end{itemize}
\end{samepage}

\begin{samepage}
    \question How long is each of the following in seconds?
      \begin{itemize}
        \item 1ms
        \vspace{5mm}
        \item 500 ms
        \vspace{5mm}
        \item 1000 ms
        \vspace{5mm}
        \item 6000 ms
        \vspace{5mm}
       \end{itemize}
\end{samepage}


\section*{\unit\_060 -- Dates and Times in Java} % Unnumbered section
\noindent \textbf{Video Length: 17:06}

\begin{samepage}
    \question What class is used to represent each of the following in modern versions of Java?
      \begin{itemize}  
        \item Dates (but not times)
        \vspace{5mm}
        \item Times (but not dates)       
        \vspace{5mm}
        \item Dates and times together
       \end{itemize}
\end{samepage}

\begin{samepage}
    \question What is a "factory method?"  When is it used?
    \vspace{5mm}
\end{samepage}
\begin{samepage}
    \question Write the code to declare a variable called "today" and retrieve the current date from the system.
    \vspace{5mm}
\end{samepage}
\begin{samepage}
    \question Assume there are two LocalDate objects called d1 and d2.  Both have valid values set.  Write the code needed to print the date that comes first on the calendar.  If the two dates are the same, you may print either date.
    \vspace{25mm}
\end{samepage}



\section*{\unit\_070 -- Swapping} % Unnumbered section
\noindent \textbf{Video Length: 5:34}

\begin{samepage}
    \question Assume that there are two String variables \texttt{\textbf{person1}} and \texttt{\textbf{person2}}. Write the code needed to "swap" the names, so that person1 now has the name that used to be in person2, and person2 now holds the name that used to be assigned to person1.  
    \vspace{5mm}
\end{samepage}

\begin{samepage}
    \question This was not covered in the video.  But you should be able to look it up in the API and answer the question.  Does the LocalDate class have a compareTo() method?  Does it work like the String compareTo method?
    \vspace{5mm}
\end{samepage}


\section*{\unit\_080 Nested If} % Unnumbered section
\noindent \textbf{Video Length: 29:06}
\begin{samepage}
    \question A big part of this video is illustrating how dates can be used and manipulated.  The main point is the nesting.  What is the basic rule about nesting (can the iff statement span into both blocks if the if statement?)
    \vspace{5mm}
\end{samepage}


\section*{\unit\_090 -- Chained If} % Unnumbered section
\noindent \textbf{Video Length: 17:17}

\begin{samepage}
    \question Write a method with the following header.  Use chained logic.  The message should be the recommended activity.
      \begin{itemize} 
        \item 40 and above message is "stay indoors"
        \item 30 and above message is "swim"
        \item 20 and above message is "walk"
        \item 10 and above message is "jog"
        \vspace{5mm}
        
       \end{itemize}
\end{samepage}

%%%%%%%%%%%%%%%%%%%%%%%%%%%%%%%%%%%%%%%%%%%%%%%%%%%%%%%%%%%%%%%%%%%%%%%%%%%%
\begin{samepage}
    \begin{center}
    \rule{0.8\textwidth}{.4pt}
    \end{center}
	\question Comment or ask questions here if there are any issues you would like for the instructor to address.  It is not necessary to answer if you do not have questions.
	\vspace{30mm}
\end{samepage}

\end{questions}

\end{document}