%!TEX program = xelatex
\documentclass[letterpaper,12pt]{exam}
\usepackage{../videoNotes}
\usepackage{xcolor}
\usepackage[dvipsnames]{xcolor}

\newcommand{\unit}{Unit 03}
\pagestyle{headandfoot}
\firstpageheader{CSC 187 \semester\ \  \unit}{}{Name: $\rule{6cm}{0.15mm}$}
\runningheader{CSC 187 \semester}{\unit}{Page \thepage\ of \numpages}
\firstpagefooter{}{}{}
\runningfooter{}{}{}

\begin{document}

\begin{questions}

\section*{\unit\_010 --- printf} % Unnumbered section

\begin{samepage}
	\question What format code does Printf use for each of the following data types?
	  \begin{itemize}
		\item \code{integers}
		\vspace{5mm}
		\item \code{real numbers}
		\vspace{5mm}
		\item \code{Strings}
		\vspace{5mm}
		\item \code{characters}
		\vspace{5mm}
	   \end{itemize}
\end{samepage}
\begin{samepage}
	\question Does the printf method automatically include a newline character?
	\vspace{5mm}
\end{samepage}
\begin{samepage}
	\question Assume there are variables \code{length} and \code{width} and a double variable named \code{area}.  Write the full System.out.printf() statement that prints the following statement.  Fill in the blanks with the length, width, and area.  Make sure you include a newline.
	\begin{verbatim}
		A rectangle with a length of ____ and a width of ____ has an area of ______.
	\end{verbatim}

	\vspace{5mm}
\end{samepage}
\section*{\unit\_020 Characters and Strings} % Unnumbered section
\begin{samepage}
	\question In Java, is a \code{char} the same as a \code{String}?  Explain the difference.
	\vspace{5mm}
\end{samepage}
\begin{samepage}
	\question In Java, what is the difference between single quotes and double quotes?  Is \code{'B'} and \code{"B"} the same?
	\vspace{5mm}
\end{samepage}

\begin{samepage}
	\question Is '\symbol{92}n' one characters or two?
	\vspace{5mm}
\end{samepage}

\begin{samepage}
	\question What are "ASCII Codes?"
	\vspace{5mm}
\end{samepage}
\begin{samepage}
	\question What are the ASCII codes less than 32 called?
	\vspace{5mm}
\end{samepage}
\begin{samepage}
	\question 
	  \begin{itemize}
		\item ' ' (space) \rule{20mm}{0.15mm}
		\vspace{5mm}
		\item '0' (zero) \rule{20mm}{0.15mm}
		\vspace{5mm}
		\item '@' \rule{20mm}{0.15mm}
		\vspace{5mm}
		\item 'A' \rule{20mm}{0.15mm} 
		\vspace{5mm}
		\item 'a' \rule{20mm}{0.15mm}
		\vspace{5mm}
		\item '~' (tilde) \rule{20mm}{0.15mm}
		\vspace{5mm}
	   \end{itemize}
\end{samepage}
\begin{samepage}
	\question What is the biggest problem with ASCII codes?
	\vspace{15mm}
\end{samepage}
\begin{samepage}
	\question How many bits does extended ASCII use?
	\vspace{5mm}
\end{samepage}


\begin{samepage}
	\question Does Java use ASCII or Unicode to represent character data?
	\vspace{5mm}
\end{samepage}

\begin{samepage}
	\question How many bits does Java use to represent type char?
	\vspace{5mm}
\end{samepage}

\begin{samepage}
	\question Is ASCII built into Unicode, or did Unicode start from scratch?
	\vspace{5mm}
\end{samepage}


%%%%%%%%%%%%%%%%%%%%%%%%%%%%%%%%%%%%%%%%%%%%%%%%%%%%%
\begin{samepage}
	\question Do you have any other questions?
	\vspace{30mm}
\end{samepage}

\end{questions}

\end{document}