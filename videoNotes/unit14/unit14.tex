
%!TEX program = xelatex
\documentclass[letterpaper,12pt]{exam}
\usepackage{../videoNotes}
\usepackage{xcolor}
\usepackage[dvipsnames]{xcolor}
\usepackage{soul}

\newcommand{\unit}{Unit 14}
\pagestyle{headandfoot}
\firstpageheader{CSC 187 \semester\ \  \unit}{}{Name: $\rule{6cm}{0.15mm}$}
\runningheader{CSC 187 \semester}{\unit}{Page \thepage\ of \numpages}
\firstpagefooter{}{}{}
\runningfooter{}{}{}

\begin{document}
\section*{\unit\_010 -- Inheritance} 
\par{\fontfamily{qzc}\selectfont\textbf{Video Length 4:10 }}
\begin{questions}

\begin{samepage}
    \question What is the key word used to extend a class?  (Be careful to get the last letter correct)
    \vspace{5mm}
\end{samepage}

\begin{samepage}
    \question Suppose you have the following line of code:
    \begin{verbatim}
        class Aaaa extends Bbbb
    \end{verbatim}
        Which is the super class, and which is the sub class?
    \vspace{5mm}
\end{samepage}
\begin{samepage}
    \question What other names are used for super and sub?
    \vspace{5mm}
\end{samepage}

\begin{samepage}
    \question What does \texttt{\textbf{final}} men when it is assigned to a class?
    \vspace{5mm}
\end{samepage}

\begin{samepage}
    \question What methods may be extended?  Which may not be extended?
    \vspace{5mm}
\end{samepage}

\begin{samepage}
    \question What is one class in the standard Java library that cannot be extended?  Why can't it be extended?
    \vspace{5mm}
\end{samepage}


\section*{\unit\_015 -- Tour of Object Methods } 
\par{\fontfamily{qzc}\selectfont\textbf{Video Length 9:20}}

\begin{samepage}
    \question What two access levels may be inherited by sub methods?
    \vspace{5mm}
\end{samepage}
\begin{samepage}
    \question What two access levels may NOT be inherited by sub methods?
    \vspace{5mm}
\end{samepage}

\begin{samepage}
    \question Explain in plain English what the .getClass() method returns.
    \vspace{5mm}
\end{samepage}

%----------------------------------

\section*{\unit\_020 -- Overriding methods } 
\par{\fontfamily{qzc}\selectfont\textbf{Video Length 6:20}}

\begin{samepage}
    \question Explain in plain English what it means to "Override" a variable?
    \vspace{5mm}
\end{samepage}

\begin{samepage}
    \question What do all of the "decorations" have in common?
    \vspace{5mm}
\end{samepage}

\begin{samepage}
    \question Does syntax require @Override in current versions of Java?
    \vspace{5mm}
\end{samepage}



%----------------------------------

\section*{\unit\_025 -- ISA and HASA Relationship } 
\par{\fontfamily{qzc}\selectfont\textbf{Video Length 6:30}}

\begin{samepage}
    \question If a subclass and a superclass exist, Is an instance of a sub class also a member of the super class?
    \vspace{5mm}
\end{samepage}

\begin{samepage}
    \question If a sub class and a super class exist, Is an instance of a super class also a member of the sub class?
    \vspace{5mm}
\end{samepage}

%----------------------------------

\section*{\unit\_030 -- Polymorphism} 
\par{\fontfamily{qzc}\selectfont\textbf{Video Length 2:50}}

\begin{samepage}
    \question What does polymorphism mean?
    \vspace{5mm}
\end{samepage}

\begin{samepage}
    \question What is "binding" in computer terminology?
    \vspace{5mm}
\end{samepage}

\begin{samepage}
    \question Does Java do early binding or late binding?
    \vspace{5mm}
\end{samepage}

%----------------------------------

\section*{\unit\_040 -- Super} 
\par{\fontfamily{qzc}\selectfont\textbf{Video Length 5:30}}

\begin{samepage}
    \question Suppose ZipCode extends PostalCode.  How would a method in the ZipCode class call the toString method in the PostalCode class?
    \vspace{5mm}
\end{samepage}

\begin{samepage}
    \question Suppose ZipCode extends PostalCode.  How would the default constructor in ZipCode call the default constructor in the PostalCode class?
    \vspace{5mm}
\end{samepage}

\begin{samepage}
    \question What is the major restriction on the placement of calls to super for constructors?  Does this restriction apply to other methods?
    \vspace{5mm}
\end{samepage}


%----------------------------------

\section*{\unit\_050 -- Practical Example} 
\par{\fontfamily{qzc}\selectfont\textbf{Video Length 19:40}}

\begin{samepage}
    \question Explain how the Customer class demonstrates that instances of the Customer class are members of the People class, but instances of the People class are not necessarily Customers.  (the setLastPurchase() method is the key)
    \vspace{25mm}
\end{samepage}

%----------------------------------

\section*{\unit\_060 -- Sorting Revisited} 
\par{\fontfamily{qzc}\selectfont\textbf{Video Length }}

\begin{samepage}
    \question What is the growth rate for QSort?
    \vspace{5mm}
\end{samepage}

\begin{samepage}
    \question (Not really covered in the video, but think about it) Why was comparable implemented only for the Person class, but not for the sub classes?
    \vspace{5mm}
\end{samepage}

\begin{samepage}
    \question Suppose we have an array called people.  Write the single command needed to sort the array if there are n elements.
    \vspace{5mm}
\end{samepage}

%----------------------------------

%%%%%%%%%%%%%%%%%%%%%%%%%%%%%%%%%%%%%%%%%%%%%%%%%%%%%
\begin{center}
    \rule{0.5\textwidth}{.4pt}
\end{center}
Please write any lingering questions you have here.
\end{questions}

\end{document}