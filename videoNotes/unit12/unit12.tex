
%!TEX program = xelatex
\documentclass[letterpaper,12pt]{exam}
\usepackage{../videoNotes}
\usepackage{xcolor}
\usepackage[dvipsnames]{xcolor}
\usepackage{soul}

\newcommand{\unit}{Unit 12}
\pagestyle{headandfoot}
\firstpageheader{CSC 187 \semester\ \  \unit}{}{Name: $\rule{6cm}{0.15mm}$}
\runningheader{CSC 187 \semester}{\unit}{Page \thepage\ of \numpages}
\firstpagefooter{}{}{}
\runningfooter{}{}{}

\begin{document}
\section*{\unit\_010 -- Terms} 
\par{\fontfamily{qzc}\selectfont\textbf{Video Length}}
\begin{questions}
\begin{samepage}
    \question What is computer hardware?
    \vspace{5mm}
\end{samepage}
\begin{samepage}
    \question What is computer software?
    \vspace{5mm}
\end{samepage}

\begin{samepage}
    \question What is machine language?
    \vspace{5mm}
\end{samepage}

\begin{samepage}
    \question What is Assembly Language?  How does it relate to machine language?
    \vspace{5mm}
\end{samepage}

\begin{samepage}
    \question What are high level languages?  Can computers immediately execute programs written in high level languages?  What needs to be done to get programs ready to execute?
    \vspace{5mm}
\end{samepage}

%----------------------------------
\section*{\unit\_020 -- Translators} 
\par{\fontfamily{qzc}\selectfont\textbf{Video Length }}

\begin{samepage}
    \question What is a compiler?
    \vspace{5mm}
\end{samepage}

\begin{samepage}
    \question What is an interpreter?
    \vspace{5mm}
\end{samepage}

\begin{samepage}
    \question This was not directly addressed, but how are compilers and interpreters similar?  How are they different?
    \vspace{15mm}
\end{samepage}

\begin{samepage}
    \question Is Java an interpreted or a compiled language?  Explain your answer.
    \vspace{15mm}
\end{samepage}


%----------------------------------
\section*{\unit\_030 -- Java Mechanics} 
\par{\fontfamily{qzc}\selectfont\textbf{Video Length }}

\begin{samepage}
    \question What does the \texttt{\textbf{javac}} program do?
    \vspace{5mm}
\end{samepage}

\begin{samepage}
    \question What does the \texttt{\textbf{java}} program do?
    \vspace{5mm}
\end{samepage}
\begin{samepage}
    \question What is in each type of file?
      \begin{itemize}
        \item .java
        \vspace{5mm}
        \item .class
        \vspace{5mm}
        \item .jar
        \vspace{5mm}
       \end{itemize}
\end{samepage}

\begin{samepage}
    \question How is a java file related to a zip file?
    \vspace{5mm}
\end{samepage}

%----------------------------------
\section*{\unit\_030 -- Designing an Object, Part 1} 
\par{\fontfamily{qzc}\selectfont\textbf{Video Length }}

\begin{samepage}
    \question What is a UML diagram?
    \vspace{5mm}
\end{samepage}

\begin{samepage}
    \question When using \texttt{\textbf{this}} is being used to call a constructor, what is the rule about where the this(...) call must be located?
    \vspace{5mm}
\end{samepage}

%----------------------------------
\section*{\unit\_020 -- Designint an Object, Part 2} 
\par{\fontfamily{qzc}\selectfont\textbf{Video Length }}

\begin{samepage}
    \question How do you get IntelliJ to generate Getters, Setters, and other standard methods for you?  Is it OK to modify those methods?
    \vspace{5mm}
\end{samepage}

%----------------------------------
\section*{\unit\_050 -- Equals And Hashcode} 
\par{\fontfamily{qzc}\selectfont\textbf{Video Length }}

\begin{samepage}
    \question What does the hascode method return?
    \vspace{5mm}
\end{samepage}

\begin{samepage}
    \question What is the relationship between a equals() and hashcode()?
    \vspace{5mm}
\end{samepage}


%----------------------------------
\section*{\unit\_060 -- Searching} 
\par{\fontfamily{qzc}\selectfont\textbf{Video Length }}

\begin{samepage}
    \question In the video, there was a boolean variable named "found."  Suppose this variable was not in the method.  Then the while statement would just be \texttt{\textbf{while(i < n)}}.  Would the method still find the smallest value?  What is the advantage of checking the found variable?
    \vspace{5mm}
\end{samepage}

%%%%%%%%%%%%%%%%%%%%%%%%%%%%%%%%%%%%%%%%%%%%%%%%%%%%%
\begin{center}
    \rule{0.5\textwidth}{.4pt}
\end{center}
Please write any lingering questions you have here.
\end{questions}

\end{document}