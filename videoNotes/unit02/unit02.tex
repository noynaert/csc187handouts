%!TEX program = xelatex
\documentclass[letterpaper,12pt]{exam}
\usepackage{../videoNotes}
\usepackage{xcolor}
\usepackage[dvipsnames]{xcolor}

\newcommand{\unit}{Unit 02}
\pagestyle{headandfoot}
\firstpageheader{CSC 187 \semester\ \  \unit}{}{Name: $\rule{6cm}{0.15mm}$}
\runningheader{CSC 187 \semester}{\unit}{Page \thepage\ of \numpages}
\firstpagefooter{}{}{}
\runningfooter{}{}{}

\begin{document}

\begin{questions}

\section*{\unit\_005 --- Comments} % Unnumbered section

\begin{samepage}
	\question In Java, what symbol is used to indicate the rest of a line is a comment?
	\vspace{5mm}
\end{samepage}

\begin{samepage}
	\question Can a program have too many comments? Briefly explain your answer.
	\vspace{15mm}
\end{samepage}

\section*{\unit\_010 Data Representation} % Unnumbered section
\begin{samepage}
	\question What is a bit?  What values can a bit store?
	\vspace{5mm}
\end{samepage}

\begin{samepage}
	\question If a system has \code{n} bits, what is the formula to tell you how many different patterns can be created?
	\vspace{5mm}
\end{samepage}

\begin{samepage}
	\question 
	  \begin{itemize}
		\item If you have 1 bit, how many patterns are possible?
		\vspace{5mm}
		\item If you have 4 bits, how many patterns are possible?
		\vspace{5mm}
		\item If you have 8 bits, how many patterns are possible?
		\vspace{5mm}
		\item If you have 10 bits, \emph{approximately} how many patterns are possible?
		\vspace{5mm}
		\item If you have 20 bits, \emph{approximately} how many patterns are possible?
		\vspace{5mm}
		\item If you have 30 bits, \emph{approximately} how many patterns are possible?
		\vspace{5mm}
		
	\end{itemize}
\end{samepage}

\begin{samepage}
	\question What is a kilobit?(approximately) What is it in terms of a power of 2?\\
	\textcolor{blue}{Hint: The answer I am looking for is about a thousand or $2^{10}$}
	\vspace{5mm}
\end{samepage}

\begin{samepage}
	\question What is a megabit? (approximately) What is it in terms of a power of 2?
	\vspace{5mm}
\end{samepage}

\begin{samepage}
	\question What is a gigabit? (approximately) What is it in terms of a power of 2?
	\vspace{5mm}
\end{samepage}

\begin{samepage}
	\question What is a terabit? (approximately) What is it in terms of a power of 2?  This was not in the video.  Guess.  Then look it up on the internet to confirm whether you were correct.
	\vspace{5mm}
\end{samepage}

\section*{\unit\_020 Integer Types} % Unnumbered section
\begin{samepage}
	\question What is a byte?
	\vspace{5mm}
\end{samepage}
\begin{samepage}
	\question What is the name given to the base 2 number system?
	\vspace{5mm}
\end{samepage}
\begin{samepage}
	\question What is the name given to the base 10 number system?
	\vspace{5mm}
\end{samepage}
\begin{samepage}
	\question What is the name given to the base 16 number system?
	\vspace{5mm}
\end{samepage}

\section*{\unit\_030 Java Integer types} % Unnumbered section
\begin{samepage}
	\question What is the largest integer that can be stored in a Java \code{byte}?
	\vspace{5mm}
\end{samepage}
\begin{samepage}
	\question What is the name of the Java data type that can hold integers up to 32,767?
	\vspace{5mm}
\end{samepage}

\begin{samepage}
	\question 32 bits can store over 4 billion values.  A Java \code{int} uses 32 bits, but it only stores numbers up to a little over 2 billion?  Why can't an int store up to 4 billion values?
	\vspace{5mm}
\end{samepage}

\hrule
\vspace{5mm}

\begin{samepage}
	\question What Java data type uses 64 bits to store integers?
	\vspace{5mm}
\end{samepage}

\section*{\unit\_040 Java Identifiers} % Unnumbered section

\begin{samepage}
	\question Create a java variable that holds the number of miles between two locations.
	\vspace{5mm}
\end{samepage}
\begin{samepage}
	\question How should the following statement be read?  \code{i = i + 1}
	\vspace{5mm}
\end{samepage}
\pagebreak
\begin{samepage}
	\question For each of the following, write "valid" next to it if it is a valid Java identifier. (hint:  if in doubt, try it in IntelliJ)  If it is invalid, say why it is invalid. Write a comment if you think it might be a style problem.
	\begin{itemize}
		\item \code{count}
		\vspace{5mm}
		\item \code{total distance}
		\vspace{5mm}
		\item \code{28}
		\vspace{5mm}
		\item \code{total.distance}
		\vspace{5mm}
		\item \code{total\_distance}
		\vspace{5mm}
		\item \code{total+distance}
		\vspace{5mm}
		\item \code{a}
		\vspace{5mm}
	   \end{itemize}
\end{samepage}
\section*{\unit\_050 Real Numbers} % Unnumbered section

\begin{samepage}
	\question What is the main data type used for real numbers in Java?
	\vspace{5mm}
\end{samepage}
\begin{samepage}
	\question Java has a second data type used for real numbers, but it is not used as often.  What is the lesser used data type for real numbers?
	\vspace{5mm}
\end{samepage}
\begin{samepage}
	\question How many significant digits are represented in a \code{double} in Java?
	\vspace{5mm}
\end{samepage}
\begin{samepage}
	\question How many significant digits are represented in a \code{float} in Java?
	\vspace{5mm}
\end{samepage}
\begin{samepage}
	\question In terms of powers of 10, what is the largest possible exponent for a \code{double}?\\
	\textcolor{blue}{I just realized I skipped this in the video.  The answer is $10^{308}$}
	\vspace{5mm}
\end{samepage}
\begin{samepage}
	\question In terms of powers of 10, what is the largest possible exponent for a \code{float}?\\
	\textcolor{blue}{I just realized I skipped this in the video.  The answer is $10^{38}$}
	\vspace{5mm}
\end{samepage}

\section*{\unit\_060 Arithmetic Operators} % Unnumbered section

\begin{samepage}
	\question What symbol is used for each operation
	  \begin{itemize}
		\item addition
		\vspace{5mm}
		\item subtraction
		\vspace{5mm}
		\item multiplication
		\vspace{5mm}
		\item division
		\vspace{5mm}
		\item modulo
		\vspace{4mm}
	   \end{itemize}
\end{samepage}

\begin{samepage}
	\question What would be printed by the following block of code?
	\begin{verbatim}
		int a = 17;
		int b = 5;
		int c = a / b;
		int d = b / a;
		int e = a % b;
		int f = b % a;
		System.out.println("c is " + c );
		System.out.println("d is " + d );
		System.out.println("e is " + e );
		System.out.println("f is " + f );
	\end{verbatim}

	\noindent c is \rule{20mm}{0.15mm}\\
	\noindent d is \rule{20mm}{0.15mm}\\
	\noindent e is \rule{20mm}{0.15mm}\\
	\noindent f is \rule{20mm}{0.15mm}\\
	
	\begin{samepage}
		\question What is the value of each of the following expressions?
		  \begin{itemize}
			\item 2 + 3 * 4
			\item 2 * 3 + 4
			\item 10 / 3
			\item 3 / 10
			\item 3 \% 10
			\item 10 \% 3			
		   \end{itemize}
	\end{samepage}
\end{samepage}




%%%%%%%%%%%%%%%%%%%%%%%%%%%%%%%%%%%%%%%%%%%%%%%%%%%%%
\begin{samepage}
	\question Do you have any other questions?
	\vspace{30mm}
\end{samepage}

\end{questions}

\end{document}