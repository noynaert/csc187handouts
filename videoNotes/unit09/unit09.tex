%!TEX program = xelatex
\documentclass[letterpaper,12pt]{exam}
\usepackage{../videoNotes}
\usepackage{xcolor}
\usepackage[dvipsnames]{xcolor}


\newcommand{\unit}{Unit 09}
\pagestyle{headandfoot}
\firstpageheader{CSC 187 \semester\ \  \unit}{}{Name: $\rule{6cm}{0.15mm}$}
\runningheader{CSC 187 \semester}{\unit}{Page \thepage\ of \numpages}
\firstpagefooter{}{}{}
\runningfooter{}{}{}

\begin{document}
\section*{\unit\_010 -- .readLine() Revisited (Part 1)} 
\par{\fontfamily{qzc}\selectfont\textbf{Video Length : 9:00}}
\begin{questions}
\begin{samepage}
    \question What does the .trim() method do?
    \vspace{5mm}
\end{samepage}

\begin{samepage}
    \question The following snippet of code counts the number of lines in a file.  There are two blanks.  Fill in the blanks in the code.
    \begin{verbatim}
         int count = \rule{20mm}{0.15mm};
         while(input.hasNextLine()){
            String line = input.nextLine();
            \rule{20mm}{0.15mm}
         }    
    \end{verbatim}
    \vspace{5mm}
\end{samepage}


\section*{\unit\_010 -- .readLine() Revisited (Part 2)} 
\par{\fontfamily{qzc}\selectfont\textbf{Video Length : 25:27}}

\begin{samepage}
    \question The term "regex" is short for \rule{50mm}{0.15mm}
    \vspace{5mm}
\end{samepage}

\begin{samepage}
    \question What is the problem with using a comma as a field separator?
    \vspace{5mm}
\end{samepage}


\begin{samepage}
    \question What does the \texttt{\textbf{String[]}} indicate?
    \vspace{5mm}
\end{samepage}

\begin{samepage}
    \question What does the split() method return?
    \vspace{5mm}
\end{samepage}

\begin{samepage}
    \question Assume there is an array of Strings called "tokens."  It was generated by the following statement.  Write a for loop that prints each of the tokens in the array.
    \begin{verbatim}
        String[] tokens = line.split(",");
    \end{verbatim}
        
    \vspace{15mm}
\end{samepage}

\begin{samepage}
    \question True or False:  Split always returns one item in the array, even if the line is empty.
    
\end{samepage}

\begin{samepage}
    \question Explain how to determine if a line is empty.
     \end{samepage}

\begin{samepage}
    \question What data type does the .split() method return?
    \vspace{5mm}
\end{samepage}

\section*{\unit\_020 -- Wrapper Classes } 
\par{\fontfamily{qzc}\selectfont\textbf{Video Length 12:15}}
\begin{samepage}
    \question What is a Wrapper Class?
    \vspace{10mm}
\end{samepage}

\begin{samepage}
    \question Whare are the two purposes that Wrapper Classes serve?
      \begin{itemize}
        \item[$\rightarrow$]
        \vspace{5mm}
        \item[$\rightarrow$]
        \vspace{5mm}
       \end{itemize}
\end{samepage}

\begin{samepage}
    \question Write the Wrapper type for each of the following primitive types
      \begin{itemize}
        \item boolean
        \vspace{5mm}
        \item char
        \vspace{5mm}
        \item byte
        \vspace{5mm}
        \item short 
        \vspace{5mm}
        \item int
        \vspace{5mm}
        \item long
        \vspace{5mm}
        \item float
        \vspace{5mm}
        \item double
        \vspace{5mm}
       \end{itemize}
\end{samepage}

\begin{samepage}
    \question What does "deprecated" mean when it appears in the Java API?
    \vspace{5mm}
\end{samepage}

\begin{samepage}
    \question In Java 17, can you instantiate objects of Wrapper classes using the \texttt{\textbf{new}} operator?  Should you?  Explain your answer.
    \vspace{5mm}
\end{samepage}

\begin{samepage}
    \question Look at the following two lines of code.  The video demonstrated that the first line did not work because we needed to make 7 into a double literal as 7.0.  Fire up Idea.  Does the declaration of the Long work?  Fix the literal so that it does work.
    \begin{verbatim}
         Double x = 7;  //An error because 7 is an integer literal, not a double literal.        
         Long big = 1234   ;  //what does it take to make this work?
        \end{verbatim}
    \vspace{5mm}
\end{samepage}


\section*{\unit\_030 -- Integer Wrapper} 
\par{\fontfamily{qzc}\selectfont\textbf{Video Length :}}
\begin{samepage}
    \question What is the constant that represents the largest int value in Java?  Write the command to print the largest Integer value.
    \vspace{5mm}
\end{samepage}

\begin{samepage}
    \question What is the constant that represents the smallest int value in Java?  Write the command to print the smallest Integer value.
    \vspace{5mm}
\end{samepage}

\begin{samepage}
    \question What is the method that Java uses to find the maximum of two integer values?  Write the command to print the larger of a and b if a and b are ints.
    \vspace{5mm}
\end{samepage}

\begin{samepage}
    \question Write a Java statement to print "+123" as an integer value.
    \vspace{5mm}
\end{samepage}
\begin{samepage}
    \question What does Java do if you call Integer.parseInt("Two")?  Note that Two is not a valid integer. 
    \vspace{5mm}
\end{samepage}


\section*{\unit\_040 -- Double Wrapper } 
\par{\fontfamily{qzc}\selectfont\textbf{Video Length :}}
\begin{samepage}
    \question What is the constant that represents the largest double value in Java?  Write the command to print the largest double value.
    \vspace{5mm}
\end{samepage}

\begin{samepage}
    \question If a number is represented in scientific notation, what constant represents the largest exponent allowed in Java.
    \vspace{5mm}
\end{samepage}


\begin{samepage}
    \question How does Java represent the concept of infinity?  How does it represent $\infty$?
    \vspace{5mm}
\end{samepage}

\begin{samepage}
    \question Can the values for POSITIVE\_INFINITY, NEGATIVE\_INFINITY and NaN be stored in regular variables in Java?
    \vspace{5mm}
\end{samepage}

\begin{samepage}
    \question What value would be stored in x if the following Java statement was executed?
    \begin{verbatim}
               x = 5.0 / 0.0;
    \end{verbatim}
    \vspace{5mm}
\end{samepage}


\begin{samepage}
    \question Write the code needed to parse a String containing a real number.  Return Double.Nan if the String cannot be parsed into a double.
    \vspace{5mm}
\end{samepage}

\section*{\unit\_045 -- Pseudo Wrapper Classes} 
\par{\fontfamily{qzc}\selectfont\textbf{No video, except for a possible upload of an in-class discussonion of this topic}}

There is a very brief video on this, but it is optional.  We will cover this in class on Monday as a special topic.  It is actually an important topic in Cyber Security.

And for the person who has been asking about prime numbers, the BigDecimal class does have some special methods related to prime numbers.

\section*{\unit\_050 -- Declaring Arrays} 
\par{\fontfamily{qzc}\selectfont\textbf{Video Length :}}

\begin{samepage}
    \question Write the code to create an array that will hold 10 values of type int.  Assume there is a constant declared called MAXIMUM\_VALUES and it is set to 10.
    \vspace{5mm}
\end{samepage}

\begin{samepage}
    \question TRUE  FALSE  In an array, all of the elements must be of the same type.
    \vspace{5mm}
\end{samepage}

\begin{samepage}
    \question How can a program determine the number of cells in an array?  Why is there no () at the end when working with arrays?
    \vspace{5mm}
\end{samepage}

\clearpage
\begin{samepage}
    \question Are arrays initialized to contain initial values?  What is the default value for the following types of arrays?  In some cases I did not demonstrate.  If you need too, write some code to figure out the answer.
    
    \begin{itemize}
        \item doubles
        \vspace{5mm}
        \item int
        \vspace{5mm}
        \item objects like Strings and Localdates
        \vspace{5mm}
        \item boolean
        \item \vspace{5mm}
        \item char
        \vspace{5mm}
    \end{itemize}
    
\end{samepage}

\section*{\unit\_060 -- Using Arrays} 
\par{\fontfamily{qzc}\selectfont\textbf{Video Length :}}
\begin{samepage}
    \question What is the difference between the logical size of an array and the physical size of the array?
    \vspace{5mm}
\end{samepage}

\begin{samepage}
    \question What is a "compact" array?
    \vspace{5mm}
\end{samepage}

\begin{samepage}
    \question What variable name is commonly understood to represent the logical size of an array?
    \vspace{5mm}
\end{samepage}

\section*{\unit\_070 -- Call by Value } 
\par{\fontfamily{qzc}\selectfont\textbf{Video Length :}}

\begin{samepage}
    \question What is the main difference between "Call by value" and "Call by reference?"
    \vspace{5mm}
\end{samepage}

\begin{samepage}
    \question Does Java use call by value or call by address?
    \vspace{5mm}
\end{samepage}

\begin{samepage}
    \question When passing primitives, what happens in the calling routine if the value of one of primitive arguments gets changed while the method is operating?
    \vspace{5mm}
\end{samepage}

\begin{samepage}
    \question Java always uses call by value.  What is stored as the "value" of an array?  (Is it the contents of the array that get stored, or is it the address where the array is stored?)
    \vspace{5mm}
\end{samepage}

\begin{samepage}
    \question How does Java simulate Call by reference when passing arrays and objects?
    \vspace{5mm}
\end{samepage}

\section*{\unit\_080 -- Common Task: Print an Array} 
\par{\fontfamily{qzc}\selectfont\textbf{Video Length :}}

\section*{\unit\_090 -- Common Task: Read from File into Array} 
\par{\fontfamily{qzc}\selectfont\textbf{Video Length :}}
\section*{\unit\_100 -- Common Task: Sum an Array and Find the Average} 
\par{\fontfamily{qzc}\selectfont\textbf{Video Length :}}
\section*{\unit\_110 -- Read a file a line at a time } 
\par{\fontfamily{qzc}\selectfont\textbf{Video Length :}}
NOTE:  This will not be in the notes.  It will be preparation for the assignment this week.




\end{questions}

\end{document}