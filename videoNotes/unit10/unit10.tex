
%!TEX program = xelatex
\documentclass[letterpaper,12pt]{exam}
\usepackage{../videoNotes}
\usepackage{xcolor}
\usepackage[dvipsnames]{xcolor}


\newcommand{\unit}{Unit 10}
\pagestyle{headandfoot}
\firstpageheader{CSC 187 \semester\ \  \unit}{}{Name: $\rule{6cm}{0.15mm}$}
\runningheader{CSC 187 \semester}{\unit}{Page \thepage\ of \numpages}
\firstpagefooter{}{}{}
\runningfooter{}{}{}

\begin{document}
\section*{\unit\_010.010 -- The Object Class} 
\par{\fontfamily{qzc}\selectfont\textbf{Video Length }}
\begin{questions}
\begin{samepage}
    \question Why is the Object class important?
    \vspace{5mm}
\end{samepage}
\begin{samepage}
    \question What does the .toString() method do?  Why is it considered a "magic method?"
    \vspace{5mm}
\end{samepage}

\begin{samepage}
    \question What data type does the .hashCode() method return?
    \vspace{5mm}
\end{samepage}

\begin{samepage}
    \question What data type does the .equals() method return?  How does the .equals() method relate to the .hashCode() method?
    \vspace{5mm}
\end{samepage}

%----------------------------------
\section*{\unit\_010.020 -- The Namespace Problem} 
\par{\fontfamily{qzc}\selectfont\textbf{Video Length }}

\begin{samepage}
    \question What is a Namespace?
    \vspace{5mm}
\end{samepage}

\begin{samepage}
    \question Why can variables have the same name in different methods?
    \vspace{5mm}
\end{samepage}

\begin{samepage}
    \question When is it OK for two methods to have the same name (and signature)?
    \vspace{5mm}
\end{samepage}

%----------------------------------
\section*{\unit\_010.030 -- Packages and Classes} 
\par{\fontfamily{qzc}\selectfont\textbf{Video Length }}

\begin{samepage}
    \question In the following url, "woz" is the name of a server.  Identify the TLD, SLD, and the subdomain.
    \begin{verbatim}
              https://woz.csmp.missouriwestern.edu
    \end{verbatim}
    \vspace{5mm}
\end{samepage}

\begin{samepage}
    \question Suppose you work for a company that used the subdomain code.ajax.com.  How would you start the package names for your company?
    \vspace{5mm}
\end{samepage}


%----------------------------------
\section*{\unit\_010.040 -- Creating a Class} 
\par{\fontfamily{qzc}\selectfont\textbf{Video Length }}

\begin{samepage}
    \question What is the base name for packages you create in this course?
    \vspace{5mm}
\end{samepage}

\begin{samepage}
    \question What file do you right-click on in order to make a package in the current project?
    \vspace{5mm}
\end{samepage}


%----------------------------------
\section*{\unit\_010.050 -- Fields and toString} 
\par{\fontfamily{qzc}\selectfont\textbf{Video Length }}

\begin{samepage}
    \question Are the variables representing the fields declared inside a method?

    \vspace{5mm}
\end{samepage}

\begin{samepage}
    \question Are the fields declared static?
    \vspace{5mm}
\end{samepage}

\begin{samepage}
    \question Is the toString() method static?
    \vspace{5mm}
\end{samepage}
\begin{samepage}
    \question What is the return type of the toString method?
    \vspace{5mm}
\end{samepage}
\begin{samepage}
    \question What are the parameters of the toString method (trick question!)
    \vspace{5mm}
\end{samepage}
\begin{samepage}
    \question Does the toString() method print?
    \vspace{5mm}
\end{samepage}
\begin{samepage}
    \question Is it normal to put a \\n at the end of the string that is returned by toString()?
    \vspace{5mm}
\end{samepage}
\begin{samepage}
    \question What String method may be used to format the output for the String
    \vspace{5mm}
\end{samepage}


%----------------------------------
\section*{\unit\_010.055 -- Users} 
\par{\fontfamily{qzc}\selectfont\textbf{Video Length }}

\begin{samepage}
    \question In the context of this course, who is the "User?"
    \vspace{5mm}
\end{samepage}

%----------------------------------
\section*{\unit\_010.060 -- Setters} 
\par{\fontfamily{qzc}\selectfont\textbf{Video Length }}

\begin{samepage}
    \question What does the reserved word "this" mean in Java?
    \vspace{5mm}
\end{samepage}

\begin{samepage}
    \question Assume that we added a field called "capital" which is a string.  Write the setter needed for the capital field.  You do not need to do any error checking.  Be sure your method meets all the criteria on the left. (The items on the left may also be the basis of test questions.)
    \begin{itemize}
        \item the setter is public
        \item the setter is NOT static
        \item the setter returns void
        \item the name starts with set
        \item the name ends with the field name
        \item the parameter is the same type as the field
        \item the parameter has the same name as the field
        \item \texttt{\textbf{this}} identifies the class variable
    \end{itemize}
    \vspace{5mm}
\end{samepage}

%----------------------------------
\section*{\unit\_010.070 -- Getters} 
\par{\fontfamily{qzc}\selectfont\textbf{Video Length }}
\begin{samepage}
    \question Assume that we added a field called "capital" which is a string.  Write the getter needed for the capital field.  Be sure your method meets all the criteria on the left. (The items on the left may also be the basis of test questions.)
    \begin{itemize}
        \item the getter is public
        \item the getter is NOT static
        \item the getter returns data type of the field
        \item the name starts with get
        \item the name ends with the field name
        \item there is no parameter
        \item the body of the method is probably a single return statement.
    \end{itemize}
    \vspace{5mm}
\end{samepage}

%----------------------------------
\section*{\unit\_010. -- } 
\par{\fontfamily{qzc}\selectfont\textbf{Video Length }}

%----------------------------------
\section*{\unit\_010. -- } 
\par{\fontfamily{qzc}\selectfont\textbf{Video Length }}

%----------------------------------
\section*{\unit\_010. -- } 
\par{\fontfamily{qzc}\selectfont\textbf{Video Length }}

%----------------------------------
\section*{\unit\_010. -- } 
\par{\fontfamily{qzc}\selectfont\textbf{Video Length }}



%%%%%%%%%%%%%%%%%%%%%%%%%%%%%%%%%%%%%%%%%%%%%%%%%%%%%
Please write any lingering questions you have here.


\end{questions}

\end{document}